%% beamer packages
% other themes: AnnArbor, Antibes, Bergen, Berkeley, Berlin, Boadilla, boxes, 
% CambridgeUS, Darmstadt, Dresden, Frankfurt, Goettingen, Hannover, Ilmenau,
%JuanLesPins, Luebeck, Madrid, Malmoe, Marburg, Montpellier, PaloAlto,
%Pittsburgh, Rochester, Singapore, Szeged, Warsaw
% other colors: albatross, beaver, crane, default, dolphin, dove, fly, lily, 
%orchid, rose, seagull, seahorse, sidebartab, structure, whale, wolverine,
%beetle

%\documentclass[xcolor=dvipsnames]{beamer}
\documentclass[table,dvipsnames]{beamer}
\usepackage{beamerthemesplit}
\usepackage{bm,amsmath,marvosym}
\usepackage{listings,color}%xcolor
\usepackage[ngerman]{babel}
\usepackage{natbib}
\usepackage[utf8]{inputenc}
\definecolor{shadecolor}{rgb}{.9, .9, .9}
\definecolor{darkblue}{rgb}{0.0,0.0,0.5}
\definecolor{myorange}{cmyk}{0,0.7,1,0}
\definecolor{mypurple}{cmyk}{0.3, 0.9, 0.0, 0.2}

% make a checkmark
\usepackage{tikz}
\def\checkmark{\tikz\fill[scale=0.4](0,.35) -- (.25,0) -- (1,.7) -- (.25,.15) -- cycle;} 

% dot product
\usetikzlibrary{arrows,positioning}
\tikzset{
    %Define standard arrow tip
    >=stealth',
    % Define arrow style
    pil/.style={->,thick}
}

% math stuff
\newcommand{\argmin}{\operatornamewithlimits{argmin}}

\lstnewenvironment{code}{
    \lstset{backgroundcolor=\color{shadecolor},
        showstringspaces=false,
        language=python,
        frame=single,
        framerule=0pt,
        keepspaces=true,
        breaklines=true,
        basicstyle=\ttfamily,
        keywordstyle=\bfseries,
        basicstyle=\ttfamily\scriptsize,
        keywordstyle=\color{blue}\ttfamily,
        stringstyle=\color{red}\ttfamily,
        commentstyle=\color{green}\ttfamily,
        columns=fullflexible
    }
}{}

\lstnewenvironment{codeout}{
    \lstset{backgroundcolor=\color{shadecolor},
        frame=single,
        framerule=0pt,
        breaklines=true,
        basicstyle=\ttfamily\scriptsize,
        columns=fullflexible
    }
}{}

\hypersetup{colorlinks = true, linkcolor=darkblue, citecolor=darkblue,urlcolor=darkblue}
\hypersetup{pdfauthor={A. Richards}, pdftitle={Projects}}

\newcommand{\rd}{\textcolor{red}}
\newcommand{\grn}{\textcolor{green}}
\newcommand{\keywd}{\textcolor{myorange}}
\newcommand{\highlt}{\textcolor{darkblue}}
\newcommand{\norm}[1]{\left\lVert#1\right\rVert}
\def\ci{\perp\!\!\!\perp}
% set beamer theme and color
\usetheme{Frankfurt}
%\usetheme{Berkeley}
\usecolortheme{orchid}
%\usecolortheme{seagull}
\setbeamertemplate{blocks}[rounded][shadow=true]

\title[Project teams]{Python and Data Science \\ (Live Coding)}
\author[Galvanize DS]{E. Wellinger, A. Richards}
\institute{}
\date[]{01.25.2017}

%%%%%%%%%%%%%%%%%%%%%%%%%%%%%%%%%%%%%%%%%%%%%%%%%%%%%%%%%%%%%%%%%%%%%%%%%%%%%%%
\begin{document}
\frame{\titlepage}
%%%%%%%%%%%%%%%%%%%%%%%%%%%%%%%%%%%%%%%%%%%%%%%%%%%%%%%%%%%%%%%%%%%%%%%%%%%%%%%
\frame{
\footnotesize
\tableofcontents
\normalsize
}

%%%%%%%%%%%%%%%%%%%%%%%%%%%%%%%%%%%%%%%%%%%%%%%%%%%%%%%%%%%%%%%%%%%%%%%%%%%%%%%
\section{Why.. Why?}
\subsection{}
%%%%%%%%%%%%%%%%%%%%%%%%%%%%%%%%%%%%%%%%%%%%%%%%%%%%%%%%%%%%%%%%%%%%%%%%%%%%%%%
\frame{   
\frametitle{Why Data Science?}
\footnotesize
\begin{block}{Well because of the data and because of the science}
 \begin{itemize}
    \item Data science means different things to different people
    \begin{itemize}
         \item Data engineering $\rightarrow$ data vis $\rightarrow$ predictive modeling
    \end{itemize}
    \item Statistics, machine-learning, databases, web-development
    \item We are in an era of unprecedented data growth
    \begin{itemize}
        \item And so few know how to effectively use data to gain insight
    \end{itemize}
    \item Science is the proponent of truth---arguably the foundation of society
    \end{itemize}
\end{block}

\visible<2->{
\begin{block}{}
\keywd{There are jobs as well!}
\end{block}
}
}

%%%%%%%%%%%%%%%%%%%%%%%%%%%%%%%%%%%%%%%%%%%%%%%%%%%%%%%%%%%%%%%%%%%%%%%%%%%%%%%
\frame{   
\frametitle{The essential data science toolkit}
\begin{block}{Subject area mastery}
 \begin{itemize}
    \item SQL and noSQL databases
    \item Associative statistics and hypothesis testing
    \item Unsupervised and supervised learning
    \item Data visualization
    \item Data products
    \end{itemize}
\end{block}

\begin{block}{Programming mastery}
 Expert level proficiency in language that is useful for data science 
\end{block}
}

%%%%%%%%%%%%%%%%%%%%%%%%%%%%%%%%%%%%%%%%%%%%%%%%%%%%%%%%%%%%%%%%%%%%%%%%%%%%%%%
\frame{   
\frametitle{Why Python?}
\begin{block}{}
\begin{itemize}
    \item There are many languages of data science (Python, R,...)
    \item Ecosystem (NumPy, matplotlib, pandas)
    \item Readability, Flexibility
    \item Glue language
    \item Object-oriented and functional
    \end{itemize}
\end{block}
}

%%%%%%%%%%%%%%%%%%%%%%%%%%%%%%%%%%%%%%%%%%%%%%%%%%%%%%%%%%%%%%%%%%%%%%%%%%%%%%%
\section{Demos}
\subsection{}

%%%%%%%%%%%%%%%%%%%%%%%%%%%%%%%%%%%%%%%%%%%%%%%%%%%%%%%%%%%%%%%%%%%%%%%%%%%%%%%
\frame{   
\frametitle{Demo 1}
\begin{block}{}
 Tutorial Example
\end{block}
}

%%%%%%%%%%%%%%%%%%%%%%%%%%%%%%%%%%%%%%%%%%%%%%%%%%%%%%%%%%%%%%%%%%%%%%%%%%%%%%%
\frame{   
\frametitle{Demo 2}
\begin{block}{}
 Advanced Example
\end{block}
}

%%%%%%%%%%%%%%%%%%%%%%%%%%%%%%%%%%%%%%%%%%%%%%%%%%%%%%%%%%%%%%%%%%%%%%%%%%%%%%%
\section{Discussion and Wrap-up}
\subsection{}

%%%%%%%%%%%%%%%%%%%%%%%%%%%%%%%%%%%%%%%%%%%%%%%%%%%%%%%%%%%%%%%%%%%%%%%%%%%%%%%
\frame{   
\frametitle{Overview}
\begin{block}{}
 \begin{enumerate}
    \item Version control based workflow
    \item Jupyter notebook
    \item Data types (lists, arrays, data frames)
    \item EDA - working with data frames, plotting
    \item Working with text (latent topics, clustering)
    \item Data visualization
    \end{enumerate}
\end{block}
}

%%%%%%%%%%%%%%%%%%%%%%%%%%%%%%%%%%%%%%%%%%%%%%%%%%%%%%%%%%%%%%%%%%%%%%%%%%%%%%%
\frame{   
\footnotesize
\begin{table}
\begin{tabular}{|c|c|}
\hline
Week& Topic \\
\hline
0   & Python workshop \\
1   & Programming and SQL \\
2   & Probability and Statistics \\
3   & Linear Models \\
4   & Supervised Learning \\
5   & NLP, Business analytics \\
6   & Unsupervised Learning \\
7   & Break Week \\
8   & Big Data / Data Engineering \\
9   & Special Topics / Case Studies \\
9   & Project \\
10  & Project \\
11  & Project \\
12  & Career services and Special topics \\
\hline
\end{tabular}
\end{table}

There are also part-time courses and free meet-ups...
}

% %%%%%%%%%%%%%%%%%%%%%%%%%%%%%%%%%%%%%%%%%%%%%%%%%%%%%%%%%%%%%%%%%%%%%%%%%%%%%%%
% \frame{   
% \frametitle{Resources and Contacts}
% \begin{block}{}
%  \begin{itemize}
%     \item 
%     \item
%     \item
%     \end{itemize}
% \end{block}
% 
% \begin{block}{blah}
%  blah blah
% \end{block}
% }

%%%%%%%%%%%%%%%%%%%%%%%%%%%%%%%%%%%%%%%%%%%%%%%%%%%%%%%%%%%%%%%%%%%%%%%%%%%%%%%
% \frame[allowframebreaks]{  
% \frametitle{References}
% \begin{tiny} \bibliography{../galvanize.bib}
% \bibliographystyle{apalike}         % Style BST file
% \end{tiny}
% }

\end{document}